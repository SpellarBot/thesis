\usepackage[T1]{fontenc}
\setmainfont[
  BoldFont = Carlito Bold,
  ItalicFont = Carlito Italic,
]{Carlito}
\usepackage{amsmath}

% erweiterte Möglichkeiten für die Positionierung von Bildern, Tabellen, ..
\usepackage{float}
%% Ermöglichen, Grafiken nebeneinander und innerhalb einer Grafik zu platzieren
\usepackage{subfig}				
%% für Tabellen
\usepackage{array}
% Urlpackage fürs korrekte formatieren von URLs
%\usepackage[hyphens]{url}
%\usepackage{hyperref}
\usepackage{xcolor}
\definecolor{lstback}{rgb}{0.94 0.94 0.94}
\definecolor{cmtcol}{rgb}{0.0 0.5 0.0}
\definecolor{keycol}{rgb}{0.0 0.0 0.5}
 
 % listings-Konfiguration
 \lstset{%
frame=single, 
aboveskip=12pt,
backgroundcolor=\color{lstback},
rulecolor=\color{lightgray},
basicstyle=\ttfamily\small,
keywordstyle=\color{keycol},
emphstyle=\bfseries,
commentstyle=\color{cmtcol},
breaklines=true,      % zu lange Zeilen umbrechen
extendedchars=true,   % Nicht-ASCII-Zeichen korrekt behandeln
literate={å}{{\r{a}}}1 {Æ}{{\AE}}1 {Å}{{\r{A}}}1 
    {ä}{{\"a}}1   {ö}{{\"o}}1 {ü}{{\"u}}1 
    {Ä}{{\"A}}1   {Ö}{{\"O}}1
    {Ü}{{\"u}}1 
    {ß}{{\ss{}}}1,
}

% Most common programming languages are already implemented for automatic
% highlighting. This is an example I (Michael Valentin Klammer) used in my
% thesis, for working with the programming language elixir. Which is rather
% uncommon.
\lstdefinelanguage{elixir}{
    morekeywords={case,catch,def,do,else,false,%
    use,alias,receive,timeout,defmacro,defp,%
    for,if,import,defmodule,defprotocol,%
    nil,defmacrop,defoverridable,defimpl,%
    super,fn,raise,true,try,end,with,%
    unless},
    otherkeywords={<-,->},
    sensitive=true,
    morecomment=[l]{\#},
    morecomment=[n]{/*}{*/},
    morestring=[b]",
    morestring=[b]',
    morestring=[b]"""
}

\lstdefinelanguage{pseudo}{
    morekeywords={if,not,then},
    sensitive=true,
    morecomment=[l]{\#},
    morecomment=[n]{/*}{*/},
    morestring=[b]",
    morestring=[b]',
    morestring=[b]"""
}
%% The following packages are recommendations by Mr. Singer who provided the
%% template for writing a thesis.

%% Farbdefinitionen
%\usepackage{color}
%% PDF
%\usepackage[unicode=true,pdfusetitle,
%bookmarks=true,bookmarksnumbered=false,bookmarksopen=false,
%breaklinks=true,pdfborder={0 0 1},backref=false,colorlinks=true]
%{hyperref}
%
%% kleinere Überschriften
%\KOMAoptions{headings=small}
%% Konfiguration des Inhaltsverzeichnisses
%\KOMAoptions{toc=listof}
%\KOMAoptions{toc=bib}
%
%% Kein Einrücken der ersten Zeile eines Absatzes, dafür
%% kleiner Abstand zwischen den Absätzen
%\setlength{\parskip}{\smallskipamount}
%\setlength{\parindent}{0pt}
%
%% keine Einrückung der Texte bei Bildunterschriften
%\usepackage{caption} 
%\captionsetup{format=plain} 
%
%\makeatletter
%
%% Alle Nummerierungen kapitelweise (chapter)
%\numberwithin{figure}{chapter}
%\numberwithin{equation}{chapter}
%\numberwithin{table}{chapter}
%
%\setcounter{secnumdepth}{2}
%\setcounter{tocdepth}{2}
%
%\makeatother
%
%\usepackage[utf8]{luainputenc}
%% Mathematikpaket
%% für die Auswahl von Schriften
%\usepackage{fontspec}
%% Sprachpakete: ngerman ist die Standardsprache
%\usepackage[english]{babel}
%% für spezifisches Layout 
%\usepackage{layout}
%% Blocksatz verbessern bei pdf Erzeugung
%\usepackage{microtype} 	
%% Kapitelüberschrift bei Querverweisen anzeigen
%\usepackage{nameref}
%% Einfügen von PDF Dateien
%\usepackage{pdfpages}
%% Anmerkungen im Text
%\usepackage{todonotes}
%% für Code
% \usepackage{listings}
